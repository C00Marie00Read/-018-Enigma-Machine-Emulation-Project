
\documentclass{article}

\begin{document}
\section{quotes from prior to ww1 book}
Monoalphabetic ciphers were used often. One example of the monoalphabetic cipher was used during the American Civil War where the author would choose a passage long enough to include all twenty-six letters, where the numbers are numbered then replace the plaintext letter by the number of that letter on page 2  \cite{Barker1}.  Other variants of the monoalphabetic cipher include James Lovell who employed on page 3 to 4\enquote{the plaintext alphabet was mixed by writing a keyword first and then adding the ramaining letters, inlcuding the ampersand (&) as a letter after Z}\cite{Barker1}.  During the nineteenth century, Thomas Jefferson on page 4, created a cipher device that \blockquote{essentially, the device was a series of twenty disks on the periphery of which mixed cipher alphabets were lettered. The disks could be arranged on a central shaft in any agreed upon order, generally according to the key.  Then by revolving each disk, the plaintext sequence could be formed.  Following this, any one of the twenty-five sequencesfound on the other lines could be set down for transmission as the ciphertext.}\cite{Barker1} Two other inventors designed a similar device after Jefferson, including a French cyptographer, Commandant Bazeries. Another common military strategy before and during the eighteenth century for encrypting messages was using a dictionary code. Using a dictionary code meant the two parties trying to communicate to each other would use a published dictionary, \enquote{the location of the desired word being indicated by the number of the page, column and line desired}\cite(Barker1).  Invisible ink codes were used also during the eighteenth century in which \enquote{Three or four lines were written in black ink would constitute the visible letter. The remaining blank parts of the sheet of paper would be filled with invisible writing, containng intelligence...}\cite{Barker1} Other \emph{secret} inks required another \enquote{chemical reagent to product the secret writing}\cite{Barker1}


Polyalphabetic Substitutions: are codes made up from several alphabet substitutions.  These types of codes developed as a result of cyptoanalysts figuring out the much simpler monoalphabetic substitutions.  
\medskip

\newpage
\begin{thebibliography}{100} %don't forget to add page numbers
\addtolength{\itemindent} {-0.2in}

\bibitem[1]{Barker1} Barker, Wayne G.\emph{The History of Codes and Ciphers in the United States Prior to World War I}. Laguna Hills, Calif: Aegean Park Press, 1979. Print.

\bibitem[2]{Barker2} Barker, Wayne G. \emph{The History of Codes and Ciphers in the United States During World War I}. 
Laguna beach, Calif: Aegean Park Press, 1979. Print.

\bibitem[3]{Barker3} Barker, Wayne G. \emph{The History of Codes and Ciphers in the United States During the Period between the World Wars, Pt. I. 1919-1929}. Laguna Hills, Calif: Aegean Park Press, 1979. Print.

\bibitem[4]{German} \emph{German Military Intelligence, 1939 -1945}. Frederick, Md: University Publications of American, 1984. Print.

\end{thebibliography}



\end{document}